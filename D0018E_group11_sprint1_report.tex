%%%%%%%%%%%%%%%%%%%%%%%%%%%%%%%%%%%%%%%%%
% Short Sectioned Assignment
% LaTeX Template
% Version 1.0 (5/5/12)
%
% This template has been downloaded from:
% http://www.LaTeXTemplates.com
%
% Original author:
% Frits Wenneker (http://www.howtotex.com)
%
% License:
% CC BY-NC-SA 3.0 (http://creativecommons.org/licenses/by-nc-sa/3.0/)
%
%%%%%%%%%%%%%%%%%%%%%%%%%%%%%%%%%%%%%%%%%

%----------------------------------------------------------------------------------------
%	PACKAGES AND OTHER DOCUMENT CONFIGURATIONS
%----------------------------------------------------------------------------------------

\documentclass[paper=a4, fontsize=11pt]{scrartcl} % A4 paper and 11pt font size

\usepackage[T1]{fontenc} % Use 8-bit encoding that has 256 glyphs
\usepackage[swedish, english]{babel} % English language/hyphenation
\usepackage{sectsty} % Allows customizing section commands

\usepackage{graphicx}

\usepackage{hyperref}

\allsectionsfont{\normalfont\scshape} % Make all sections centered, the default font and small caps
\sectionfont{\center \normalfont\scshape}

\usepackage{fancyhdr} % Custom headers and footers
\pagestyle{fancyplain} % Makes all pages in the document conform to the custom headers and footers
\fancyhead{} % No page header - if you want one, create it in the same way as the footers below
\fancyfoot[L]{} % Empty left footer
\fancyfoot[C]{} % Empty center footer
\fancyfoot[R]{\thepage} % Page numbering for right footer
\renewcommand{\headrulewidth}{0pt} % Remove header underlines
\renewcommand{\footrulewidth}{0pt} % Remove footer underlines

\setlength{\headheight}{13.6pt} % Customize the height of the header
\setlength\parindent{0pt} % Removes all indentation from paragraphs - comment this line for an assignment with lots of text

%----------------------------------------------------------------------------------------
%	TITLE SECTION
%----------------------------------------------------------------------------------------

\newcommand{\horrule}[1]{\rule{\linewidth}{#1}} % Create horizontal rule command with 1 argument of height

\title{	
\normalfont \normalsize 
\textsc{Lule� Tekniska Universitet} \\ [25pt] % Your university, school and/or department name(s)
\horrule{1pt} \\[0.4cm] % top horizontal rule
\huge Sprints \\ % The assignment title
\horrule{1pt} \\[0.5cm] % bottom horizontal rule
}

\author{Victor Persson,\\ Niklas Eliasson} % Your name

\date{\normalsize\today} % Today's date or a custom date

\begin{document}

\maketitle % Print the title

%----------------------------------------------------------------------------------------

\section{}

%------------------------------------------------

%\subsection{Summary}

\subsection{User stories}
	Anv�ndare
	\begin{itemize}
		\item Butiksadministrat�r
		\begin{itemize}
			\item r/w priser
			\item r/w kampanjer
			\item r/w lagerstatus
			\item r/w kategorier
			\item r leveransstatus
			\item r kundinfo
		\end{itemize}
		\item Lagerarbetare
		\begin{itemize}
			\item r/w lagerstatus
			\item r/w leveransstatus
			\item r kundinfo
		\end{itemize}
		\item Inloggad kund
		\begin{itemize}
			\item r/w sin egen kontaktinformation
			\item l�sa sin egen orderhistorik
			\item l�sa sortimentet (produkter, priser, kampanjer, lagerstatus)
			\item l�gga ordrar
			\item ? Spara/skicka kundkorg
		\end{itemize}
		\item Ej inloggad kund
		\begin{itemize}
			\item l�sa sortimentet (produkter, priser, kampanjer, lagerstatus)
			\item l�gga ordrar
			\item ? skicka kundkorg (e-post)
		\end{itemize}

	\begin{figure}
		\includegraphics[with=\linewidth]{artifacts/Inventory+Users.jpeg}
		\caption{}
		\label{fig:1}
	\end{figure}
	
	\begin{figure}
		\includegraphics[with=\linewidth]{artifacts/Lager.jpeg}
		\caption{}
		\label{fig:2}
	\end{figure}

	\begin{figure}
		\includegraphics[with=\linewidth]{artifacts/ButiksAdmin.jpeg}
		\caption{}
		\label{fig:3}
	\end{figure}

	\begin{figure}
		\includegraphics[with=\linewidth]{artifacts/Admin.jpeg}
		\caption{}
		\label{fig:4}
	\end{figure}

\subsection{System architecture}
	We use a MariaDB database and a NGINX webserver with Phusion Passenger for Ruby on Rails. We host the servers ourselves because it seemed seemed fun and fairly simple. 

\subsection{Backlog}
	Planing is done at Trello.com
	\url{https://trello.com/b/JxDCHBcm}

	\begin{figure}
		\includegraphics[with=\linewidth]{artifacts/Trello_Sprint1.png}
		\caption{}
		\label{fig:5}
	\end{figure}

\subsection{Database schema}
	\begin{figure}
		\includegraphics[with=\linewidth]{artifacts/DB-Design.jpg}
		\caption{Database design. Minor changes might have been made. See original paper copy for updated version.}
		\label{fig:6}
	\end{figure}

\subsection{Code}
	All code is avalible at github.
	\url{https://github.com/nikalas/D0018E-Databasteknik.git}

%\subsection{Test case specifications}

\subsection{Limitations and improvements}

	We decided to put off saving payment methods and/or information. Might end up readding it to the backlog if it looks like we will have time to spare. 

\end{document}
