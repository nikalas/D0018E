%%%%%%%%%%%%%%%%%%%%%%%%%%%%%%%%%%%%%%%%%
%  Sectioned Assignment
% LaTeX Template
% Version 1.0 (5/5/12)
%
% This template has been downloaded from:
% http://www.LaTeXTemplates.com
%
% Original author:
% Frits Wenneker (http://www.howtotex.com)
%
% License:
% CC BY-NC-SA 3.0 (http://creativecommons.org/licenses/by-nc-sa/3.0/)
%
%%%%%%%%%%%%%%%%%%%%%%%%%%%%%%%%%%%%%%%%%

%------------------------------------------------------------------------
%PACKAGES AND OTHER DOCUMENT CONFIGURATIONS
%------------------------------------------------------------------------

\documentclass[paper=a4, fontsize=11pt]{report} % A4 paper and 11pt font size

\usepackage[swedish, english]{babel} % Swedish and English language/hyphenation
\usepackage[T1]{fontenc} % Use 8-bit encoding that has 256 glyphs
\usepackage[a4paper]{geometry}
\usepackage{hyperref}
\usepackage[myheadings]{fullpage}
\usepackage{fancyhdr}
\usepackage{lastpage}
\usepackage{graphicx, wrapfig, subcaption, setspace, booktabs}
\usepackage[T1]{fontenc}
\usepackage[font=small, labelfont=bf]{caption}
\usepackage{fourier}
\usepackage[protrusion=true, expansion=true]{microtype}
\usepackage{sectsty}
\usepackage{url, lipsum}
\usepackage{amsmath}
\usepackage{float}
\usepackage{lmodern}
\usepackage[normalem]{ulem}
\useunder{\uline}{\ul}{}

%------------------------------------------------------------------------
%TITLE SECTION
%------------------------------------------------------------------------

\newcommand{\horrule}[1]{\rule{\linewidth}{#1}} % Create horizontal rule command with 1 argument of height
\onehalfspacing
\setcounter{tocdepth}{5}
\setcounter{secnumdepth}{5}
\usepackage{titlesec}

\pagestyle{fancy}
\fancyhf{}
\setlength\headheight{15pt}
\fancyhead[L]{Niklas Eliasson, Victor Persson}
\fancyhead[R]{Luleå Tekniska Universitet}
\fancyfoot[C]{\thepage}

\begin{document}

\title{
	\normalfont \normalsize
	\textsc{Luleå Tekniska Universitet} \\ [25pt] % Your university, school and/or department name(s)
	\horrule{1pt} \\[0.4cm] % top horizontal rule
	\huge Sprint 4 \\ % The assignment title
	\horrule{1pt} \\[0.5cm] % bottom horizontal rule
}

\author{Victor Persson,\\ Niklas Eliasson} % Your name

\date{\normalsize\today} % Today's date or a custom date

\maketitle % Print the title

\tableofcontents
\thispagestyle{empty}
\sectionfont{\scshape}
%
\newpage
\setcounter{page}{1}

%------------------------------------------------

\section*{Summary}
\addcontentsline{toc}{section}{Summary}
We are working on a e-comerce site for selling patches and accesories such as
belts and zippers for student overalls. It is intended to be dynamic with a
fully functional content management system. \\
For educational purposes Ruby on Rails was chosen. \\
Scrum planing was done at Trello.com and GitHub.com was used as VCS. \\


\section*{User stories}
\addcontentsline{toc}{section}{User stories}
Användare
\begin{itemize}
	Sectioned\item Butiksadministratör
		Sectioned\begin{itemize}
			Assignment\item r/w priser
			Assignment\item r/w kampanjer
			Assignment\item r/w lagerstatus
			Assignment\item r/w kategorier
			Assignment\item r leveransstatus
			Assignment\item r kundinfo
		Sectioned\end{itemize}
	Sectioned\item Lagerarbetare
		Sectioned\begin{itemize}
			Assignment\item r/w lagerstatus
			Assignment\item r/w leveransstatus
			Assignment\item r kundinfo
		Sectioned\end{itemize}
	Sectioned\item Inloggad kund
		Sectioned\begin{itemize}
			Assignment\item r/w sin egen kontaktinformation
			Assignment\item läsa sin egen orderhistorik
			Assignment\item läsa sortimentet (produkter, priser, kampanjer, lagerstatus)
			Assignment\item lägga ordrar
			Assignment\item ? Spara/skicka kundkorg
		Sectioned\end{itemize}

		See Figure \ref{fig:2} - \ref{fig:4}

	\begin{figure}
		Sectioned\includegraphics[scale=0.12]{artifacts/Lager.jpeg}
		Sectioned\caption{}
		Sectioned\label{fig:2}
	\end{figure}

	\begin{figure}
		Sectioned\includegraphics[scale=0.12]{artifacts/ButiksAdmin.jpeg}
		Sectioned\caption{}
		Sectioned\label{fig:3}
	\end{figure}

	\begin{figure}
		Sectioned\includegraphics[scale=0.12]{artifacts/Admin.jpeg}
		Sectioned\caption{}
		Sectioned\label{fig:4}
	\end{figure}

\section*{System architecture}
\addcontentsline{toc}{section}{System architecture}
	During development we run the system on Ruby on Rails (RoR) built in web server
	Puma and sqlight3 for simplicity, but intend to move to a MariaDB database
	and a NGINX webserver with Phusion Passenger for RoR. We host
	the servers ourselves because it seemed seemed fun, educational and fairly simple.

\section*{Backlog}
\addcontentsline{toc}{section}{backlog}

	\begin{tabular}{|l|l|l|l|l|}
		Sectioned\hline
		Sectioned \#  & Sprint 4                      & Priority & Time est. & Description \\ \hline
		Sectioned 301 & Startsida                     & 100      & 2         &             \\ \hline
		Sectioned 302 & Dynamisk meny från kategorier & 90       & 2         &             \\ \hline
		Sectioned 304 & Produktsida                   & 110      & 4         &             \\ \hline
		Sectioned 305 & Kundkorg                      & 80       & 8         &             \\ \hline
		Sectioned 307 & Profilsida (kundkort)         & 45       & 2         &             \\ \hline
		Sectioned 308 & Orderhistorik                 & 45       & 1         &             \\ \hline
		Sectioned 309 & Kundinlogg                    & 50       & 2         &             \\ \hline
		Sectioned 310 & Registrering                  & 60       & 4         &             \\ \hline
		Sectioned 312 & kommentarer/betygsättning     & 70       & 8         &             \\ \hline
		Sectioned 313 & Produktkategorier             & 75       & 2         &             \\ \hline
		Sectioned 400 & Backendinloggning             & 25       & 2         &             \\ \hline
		Sectioned 1   & Personnummer -/+ hantering    & 5        & 1         &             \\ \hline
	\end{tabular}

	\begin{tabular}{|l|l|l|l|l|}
		Sectioned \hline
		Sectioned \#  & Left in backlog  & Priority & Time est. & Description \\ \hline
		Sectioned 410 & Kampanjhantering & 10       & 3         &             \\ \hline
		Sectioned 311 & Produktsökning   & 45       & 1         &             \\ \hline
		Sectioned 306 & Betalsida        & 5        & 12        &             \\ \hline
	\end{tabular}


	Planing is done at Trello.com
	\url{https://trello.com/b/JxDCHBcm}\
	This is just a small section of the backlog. For history of all sprints and deeper
	explanation of the backlog items, refer to Trello.

\section*{Database schema}
\addcontentsline{toc}{section}{Database schema}
See Figure \ref{fig:6}
\begin{figure}
	Sectioned\includegraphics[scale=0.7]{artifacts/db_implemented_1_3.png}
	Sectioned\caption{Database design.}
	Sectioned\label{fig:6}
\end{figure}

\section*{Code}
\addcontentsline{toc}{section}{Code}
All code is avalible at github.
\url{https://github.com/nikalas/D0018E-Databasteknik.git}

\section*{Test case specifications}
\addcontentsline{toc}{section}{Test case specifications}

	Problem: Item out of stock? \
		A customer adds a product to the basket. If the product goes out of
		stock before checkout, how is this handled?

	Solution: \
		At checkout a check is made if the product is still in stock. If not
		the custumer is brought back to the 'carts' page and asked to remove
		the product that is no longer available and that the cart has to be
		updated.

\section*{Limitations and improvements}
\addcontentsline{toc}{section}{Limitations and improvements}

	We decided to put off saving payment methods and/or information. Might
	end up readding it to the backlog if it looks like we will have time
	to spare. Non-registerd customers have also been put on hold for now.
	Products search, sorting, sale campaigns, and uploading pictures
	through the backend has also been put on hold, since we didn't have
	time to fully implement them.

\end{document}
